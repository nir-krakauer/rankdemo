\documentclass[]{article}

\usepackage[
    type={CC},
    modifier={by-sa},
    version={4.0},
]{doclicense}



%=================================================================
\begin{document}

\title{An importance ranking in a network of dependencies}
\author{Nir Y. Krakauer}
\date{\today}
\maketitle
\doclicenseThis

%%%%%%%%%%%%%%%%%%%%%%%%%%%%%%%%%%%%%%%%%%

Consider a set of $N$ programs indexed 1 to $N$, where some programs depend on others (directly or indirectly (transitively); a program is never a dependency of itself). This is described by the $N \times N$ boolean dependency matrix (adjacency matrix of the corresponding directed graph) $\mathbf{D}$, where $d_{i, j}$ is 1 if program $j$ depends on program $i$ and 0 otherwise. Assume each program has intrinsic value $v_i$. We can find a program importance $p_i$ that is the sum of its own value and the importance of all programs which depend on it by solving the linear system

$$\mathbf{D}\mathbf{p} + \mathbf{v} = \mathbf{p},$$

which can be rewritten as

$$(\mathbf{I} - \mathbf{D})\mathbf{p} = \mathbf{v}$$

where $\mathbf{I}$ is the $N \times N$ identity matrix.

This is the same formalism as PageRank, except that there the intrinsic values are all zero and each $d_{i, j}$ is a transition probability from page $j$ to page $i$.

If a partial dependency matrix is available that has at least all the direct dependencies marked, it can be filled out by applying a transitive closure algorithm, most simply by applying (until a full pass finds nothing more to fill in) the rule that for all $i, j, k$, if indices $(i, j)$ and $(k, i)$ are dependencies, so is $(k, j)$. This simple rule has order $N^3$ computational cost though, so more efficient implementations should be considered to bring the cost down to order $N^2$ where the number of programs considered is large.

Assuming (reasonably) no codependency, the programs and their transitive dependencies form a directed acyclic graph, and a theorem states that we can order the programs such that $i$ is a dependency of program $j$ only if $j > i$ (a so-called topological sort). This would make the matrix $\mathbf{D}$, and with it $\mathbf{I} - \mathbf{D}$, (upper) triangular, so that we can solve for $\mathbf{p}$ in at most $N^2$ arithmetic operations.


\end{document}

